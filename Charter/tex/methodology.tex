Our solution involves creating a laser-based musical instrument, which we refer to as the "Laser Harp." The Laser Harp consists of a series of vertical laser beams, each representing a different note or sound. At the top of the setup, corresponding photo transistors detect the presence of these laser beams. When a student interrupts a laser beam by blocking it with their hand or an object, the photo transistor detects the change and sends a signal to a central processing unit, such as a Raspberry Pi. This signal triggers the playback of a specific sound associated with the interrupted laser beam.

By interacting with the Laser Harp, students can see immediate cause-and-effect relationships between their actions and the system's response. This hands-on experience helps to demystify abstract STEM concepts and demonstrates how technology can be used creatively and practically.
