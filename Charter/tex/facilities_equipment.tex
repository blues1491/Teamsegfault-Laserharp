Facilities and equipment needed for the successfully finish of the project are varied yet much needed. We would need a well equipped laboratory area that serves as the main area where the project will be put together, researched and tested. This space will be the main environment needed to work on the project and have collaborative work with others on the group. These facilities can be found all over UTA including the ERB, furthermore in terms of tools. We would need the essential tools including a soldering iron, good quality of solder and flux in order to create a secure connection and integrity of electronic assembling. Along side these tools a area to store our project and the components is needed, This will all be provided by UTA in the senior design area. More tools needed are the oscilloscope as it can provide a insight to the voltage of the circuit and necessary signal of the voltage. The oscilloscope can also help troubleshoot issues within the wiring by viewing these voltages. This equipment can be bought or borrowed from UTA and there corresponding labs. The last equipment needed is the 3d printer in order to create a frame and test different types of geometry for the frame. This equipment is borrow able within the lab area already provided or can be used in the maker space. In summary, all these facilities and equipment's form the backbone of the development of the project. these tools are assets that will help us achieve milestones efficiently and deliver a solution to the stakeholders and product owner
