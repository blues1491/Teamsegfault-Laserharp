\section{Security Requirements}

The security requirements for the laser harp project ensure that all aspects of information security and privacy are addressed. This includes the implementation of encryption standards, secure data storage procedures, robust authentication mechanisms, and strong password policies to protect user data and system integrity.

\subsection{Data Encryption}
\subsubsection{Description}
All sensitive data transmitted and stored by the laser harp system must be encrypted using industry-standard encryption algorithms. This includes user settings, calibration data, and any other personal information.
\subsubsection{Source}
Development Team
\subsubsection{Constraints}
Encryption algorithms must be efficient to not degrade system performance. Encryption keys must be securely managed and stored.
\subsubsection{Standards}
AES (Advanced Encryption Standard) - NIST FIPS 197
\subsubsection{Priority}
Critical


\subsection{Secure Data Storage}
\subsubsection{Description}
All data stored on the laser harp system must be secured to prevent unauthorized access. This includes the use of secure storage mechanisms and regular data backups.
\subsubsection{Source}
Development Team
\subsubsection{Constraints}
Secure storage solutions must be cost-effective and reliable. Data backups must be performed regularly and stored securely.
\subsubsection{Standards}
ISO/IEC 27001:2013 - Information security management systems
\subsubsection{Priority}
High


\subsection{Secure Firmware Updates}
\subsubsection{Description}
Firmware updates for the laser harp system must be delivered securely to prevent unauthorized modifications. This includes the use of digital signatures to verify the integrity and authenticity of the firmware.
\subsubsection{Source}
Development Team
\subsubsection{Constraints}
Firmware update mechanisms must be reliable and user-friendly to ensure users can easily apply updates.
\subsubsection{Standards}
NIST SP 800-147 - BIOS Protection Guidelines
\subsubsection{Priority}
Moderate
