
The Laser Harp project is a creative fusion of music and technology, designed to simulate the playing of a musical harp using laser beams and phototransistors. The concept revolves around interrupting laser beams to trigger audio files corresponding to specific notes, providing a unique and interactive way to explore music and STEM concepts. This product is aimed at engaging users, particularly students, by showcasing how digital technology can be used in artistic and educational contexts.

The detailed design presented in this document builds upon the foundational work outlined in the Requirement Specification and Architectural Design documents. These documents provide comprehensive background material, detailing the functional requirements and architectural decisions that shaped the development of the Laser Harp. This document aims to delve into the specific design choices, algorithms, and interfaces implemented to meet the project's objectives. It serves as a critical guide for understanding the technical intricacies involved in bringing the Laser Harp from concept to reality.