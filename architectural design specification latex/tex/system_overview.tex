This section describes the overall structure of the laser harp software system. The architecture is designed in layers, with each layer comprising related elements of similar capabilities. These layers are highly independent of each other but have clearly defined interfaces and interactions. Each layer is identified individually and is unique in its function and purpose within the system. The high-level block diagram below illustrates the architectural layers and their interactions.

\begin{figure}[h!]
\centering
\includegraphics[width=0.60\textwidth]{images/Layers}
\caption{A simple architectural layer diagram}
\end{figure}

\subsection{Input Layer Description}

The input layer allows users to interact with the laser harp hardware, triggering sounds and changing settings via various input mechanisms such as laser diodes, dials, buttons, and a touchscreen. This layer captures user actions and sends corresponding signals to the control unit.

\subsection{Control Unit Layer Description}

The control unit layer processes input data, manages system logic, and generates output signals. It interprets signals from the input layer and translates them into commands for the output layer. This layer is responsible for managing sound settings, user settings, and coordinating the overall functionality of the laser harp.

\subsection{Output Layer Description}

The output layer is responsible for providing feedback to the user through visual and audio channels. This includes producing the appropriate sounds based on user input and displaying relevant information on the touchscreen. The output layer ensures that the laser harp's responses are immediate and accurately reflect user interactions
